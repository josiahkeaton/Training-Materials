\documentclass[hyperref={bookmarks=false},aspectratio=169]{beamer}
\usepackage[utf8]{inputenc}
\usepackage{standalone}
% ---------------  Define theme and color scheme  -----------------
\usetheme[sidebarleft]{CEIT}  % 3 options: minimal, sidebarleft, sidebarright
\usepackage{fontspec}
\newfontfamily{\ps} [ Path = ./fonts/]{TypeWheel-Regular.ttf}

%\setbeamertemplate{footline}[frame number]
\setbeamertemplate{footline}[text line]{%
	\parbox{\linewidth}{\vspace*{-8pt} \textcopyright \  Centre for Development of Advanced Computing (C-DAC)  \hfill \insertpagenumber}}
\setbeamertemplate{navigation symbols}{}
% ------------  Information on the title page  --------------------
\title[PPT Title]
{\bfseries{Database Concepts}}

\subtitle{Lab Session 1}

\author[Jishnu T U] %\& Managed]
{Jishnu T U\inst{1} } %\and Managed\inst{2}}

\institute[CEIT]
{
  \inst{1}
  Trainer\\
  Centre of Excellence in IT,PNG
 % \and
 % \inst{2}
 % Trainer\\
 % Centre of Excellence in IT,PNG
}

\date[CEIT, 2014]
{Centre of Excellence in IT,PNG October 2019}
%------------------------------------------------------------

%------------------------------------------------------------
%The next block of commands puts the table of contents at the 
%beginning of each section and highlights the current section:

\AtBeginSection[]
{
  \begin{frame}
    \frametitle{Table of Contents}
    \tableofcontents[currentsection]
  \end{frame}
}

%------------------------------------------------------------


\begin{document}

\frame{\titlepage}  % Creates title page

%---------   table of contents after title page  ------------
\begin{frame}
\frametitle{Table of Contents}
\tableofcontents
\end{frame}
%---------------------------------------------------------


\section{Introduction to Oracle}
\begin{frame}

	\frametitle{ Introduction to Oracle}
	
	\begin{itemize}
	
	    \item
	
	    \item
	
	\end{itemize}

\end{frame}

\section{ SQL* Plus}

\begin{frame}

	\frametitle{ SQL* Plus}
	
	\begin{itemize}
	
	    \item
	
	    \item
	
	\end{itemize}
	
\end{frame}
\section{ DDL Commands}

\begin{frame}
	\frametitle{ DDL Commands}
	
	\begin{itemize}
	
	    \item
	
	    \item
	
	\end{itemize}
	
\end{frame}
\begin{frame}
	\frametitle{ DDL Commands}
	
	\begin{itemize}
	
	    \item<1->  \textbf{Problem :}
	    
	     Create a table 'students' with columns student\_id, first\_name, last\_name, email\_id, dob ?
	
	    \item<2->   
	    {\ps
	    	
	    	
	    	 CREATE TABLE students 
	    	
	    	(student\_id INTEGER AUTO\_INCREMENT PRIMARY KEY, 
	    	
	    	first\_name VARCHAR(50), 
	    	
	    	last\_name VARCHAR(50), 
	    	
	    	email\_id VARCHAR(255),  
	    	
	    	dob DATE);
	    	
	    	
	    }
	
	\end{itemize}
	
\end{frame}
\section{DML  Commands}
\begin{frame}

	\frametitle{ DML  Commands}
	
	\begin{itemize}
	
	    \item
	
	    \item
	
	\end{itemize}
	
\end{frame}
\section{ DCL Commands}
\begin{frame}

	\frametitle{  DCL Commands}
	
	\begin{itemize}
	
	    \item
	
	    \item
	
	\end{itemize}
	
\end{frame}

\end{document}
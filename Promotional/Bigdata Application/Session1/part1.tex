\section{Introduction to BigData}

There is no place where Big Data does not exist!

 The curiosity about what is Big Data has been soaring in the past few years. Let me tell you some mind-boggling facts! Forbes reports that every minute, users watch 4.15 million YouTube videos, send 456,000 tweets on Twitter, post 46,740 photos on Instagram and there are 510,000 comments posted and 293,000 statuses updated on Facebook!

Just imagine the huge chunk of data that is produced with such activities. This constant creation of data using social media, business applications, telecom and various other domains is leading to the formation of Big Data.

 Big Data refers to the large amounts of data which is pouring in from various data sources and has different formats.
Big Data is much more than a collection of datasets with different formats, it is an important asset which can be used to obtain enumerable benefits.

  The major sources of Big Data are social media sites, sensor networks, digital images/videos, cell phones, purchase transaction records, web logs, medical records, archives, military surveillance, eCommerce, complex scientific research and so on. All these information amounts to around some Quintillion bytes of data. By 2020, the data volumes will be around 40 Zettabytes which is equivalent to adding every single grain of sand on the planet multiplied by seventy-five.
  
  \section{Big Data Analytics}
  Big data analytics examines large and different types of data to uncover hidden patterns, correlations and other insights
  
  
  Now that I have told you what is Big Data and how it’s being generated exponentially, let me present to you a very interesting example of how Starbucks, one of the leading coffeehouse chain is making use of this Big Data.

I came across this article by Forbes which reported how Starbucks made use of Big Data to analyse the preferences of their customers to enhance and personalize their experience. They analysed their member’s coffee buying habits along with their preferred drinks to what time of day they are usually ordering. So, even when people visit a “new” Starbucks location, that store’s point-of-sale system is able to identify the customer through their smartphone and give the barista their preferred order. In addition, based on ordering preferences, their app will suggest new products that the customers might be interested in trying. This my friends is what we call Big Data Analytics.

Basically, Big Data Analytics is largely used by companies to facilitate their growth and development. This majorly involves applying various data mining algorithms on the given set of data, which will then aid them in better decision making.

\subsection{Big Data Applications}

These are some of the following domains where Big Data Applications has been revolutionized:
\begin{enumerate}
    \item Entertainment: Netflix and Amazon use Big Data to make shows and movie recommendations to their users.
    \item Insurance: Uses Big data to predict illness, accidents and price their products accordingly.
    \item Driver-less Cars: Google’s driver-less cars collect about one gigabyte of data per second. These experiments require more and more data for their successful execution.
    \item  Education: Opting for big data powered technology as a learning tool instead of traditional lecture methods, which enhanced the learning of students as well aided the teacher to track their performance better.
    \item Automobile: Rolls Royce has embraced Big Data by fitting hundreds of sensors into its engines and propulsion systems, which record every tiny detail about their operation. The changes in data in real-time are reported to engineers who will decide the best course of action such as scheduling maintenance or dispatching engineering teams should the problem require it.
    \item Government: A very interesting use of Big Data is in the field of politics to analyse patterns and influence election results. Cambridge Analytica Ltd. is one such organisation which completely drives on data to change audience behaviour and plays a major role in the electoral process.
    \item Smarter Healthcare: Making use of the petabytes of patient’s data, the organization can extract meaningful information and then build applications that can predict the patient’s deteriorating condition in advance.
    \item Telecom: Telecom sectors collects information, analyzes it and provide solutions to different problems. By using Big Data applications, telecom companies have been able to significantly reduce data packet loss, which occurs when networks are overloaded, and thus, providing a seamless connection to their customers.
    \item Retail: Retail has some of the tightest margins, and is one of the greatest beneficiaries of big data. The beauty of using big data in retail is to understand consumer behavior. Amazon’s recommendation engine provides suggestion based on the browsing history of the consumer.
    \item Traffic control: Traffic congestion is a major challenge for many cities globally. Effective use of data and sensors will be key to managing traffic better as cities become increasingly densely populated.
    \item Manufacturing: Analyzing big data in the manufacturing industry can reduce component defects, improve product quality, increase efficiency, and save time and money.
    \item Search Quality: Every time we are extracting information from google, we are simultaneously generating data for it. Google stores this data and uses it to improve its search quality.
    \item Finance: Banks and financial services firms use analytics to differentiate fraudulent interactions from legitimate business transactions. The analytics systems suggest immediate actions, such as blocking irregular transactions, which stops fraud before it occurs and improves profitability. 
\end{enumerate}









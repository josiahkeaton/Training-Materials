
\section{Big Data Applications}

The primary goal of Big Data applications is to help companies make more informative business decisions by analyzing large volumes of data. It could include web server logs, Internet click stream data, social media content and activity reports, text from customer emails, mobile phone call details and machine data captured by multiple sensors

Organisations from different domain are investing in Big Data applications, for examining large data sets to uncover all hidden patterns, unknown correlations, market trends, customer preferences and other useful business information.
\subsection{Healthcare}

The level of data generated within healthcare systems is not trivial. Traditionally, the health care industry lagged in using Big Data, because of limited ability to standardize and consolidate data.

But now Big data analytics have improved healthcare by providing personalized medicine and prescriptive analytics. Researchers are mining the data to see what treatments are more effective for particular conditions, identify patterns related to drug side effects, and gains other important information that can help patients and reduce costs.

With the added adoption of mHealth, eHealth and wearable technologies the volume of data is increasing at an exponential rate. This includes electronic health record data, imaging data, patient generated data, sensor data, and other forms of data.


By mapping healthcare data with geographical data sets, it’s possible to predict disease that will escalate in specific areas. Based of predictions, it’s easier to strategize diagnostics and plan for stocking serums and vaccines.


\subsection{Manufacturing}
Predictive manufacturing provides near-zero downtime and transparency. It requires an enormous amount of data and advanced prediction tools for a systematic process of data into useful information.


\subsection{Media \& Entertainment}
Various companies in the media and entertainment industry are facing new business models, for the way they –  create, market and distribute their content. This is happening because of current consumer’s search and the requirement of accessing content anywhere, any time, on any device.

Big Data provides actionable points of information about millions of individuals. Now, publishing environments are tailoring advertisements and content to appeal consumers. These insights are gathered through various data-mining activities. Big Data applications benefits media and entertainment industry by:
\begin{itemize}
    \item Predicting what the audience wants
    \item Scheduling optimization
    \item Increasing acquisition and retention
    \item Ad targeting
    \item Content monetization and new product development
\end{itemize}

\subsection{Internet of Things (IoT)}
Data extracted from IoT devices provides a mapping of device inter-connectivity. Such mappings have been used by various companies and governments to increase efficiency. IoT is also increasingly adopted as a means of gathering sensory data, and this sensory data is used in medical and manufacturing contexts

\subsection{Government}
The use and adoption of Big Data within governmental processes allows efficiencies in terms of cost, productivity, and innovation. In government use cases, the same data sets are often applied across multiple applications \& it requires multiple departments to work in collaboration.
Since Government majorly acts in all the domains, thus it plays an important role in innovating Big Data applications in each and every domain. Let me address some of the major areas:

Cyber security \& Intelligence
The federal government launched a cyber security research and development plan that relies on the ability to analyze large data sets in order to improve the security of U.S. computer networks.

The National Geospatial-Intelligence Agency is creating a “Map of the World” that can gather and analyze data from a wide variety of sources such as satellite and social media data. It contains a variety of data from classified, unclassified, and top-secret networks.

Crime Prediction and Prevention
Police departments can leverage advanced, real-time analytics to provide actionable intelligence that can be used to understand criminal behaviour, identify crime/incident patterns, and uncover location-based threats.

Pharmaceutical Drug Evaluation
According to a McKinsey report, Big Data technologies could reduce research and development costs for pharmaceutical makers by $40 billion to $70 billion. The FDA and NIH use Big Data technologies to access large amounts of data to evaluate drugs and treatment.

Scientific Research
The National Science Foundation has initiated a long-term plan to:

Implement new methods for deriving knowledge from data
Develop new approaches to education
Create a new infrastructure to “manage, curate, and serve data to communities”.

Weather Forecasting
The NOAA (National Oceanic and Atmospheric Administration) gathers data every minute of every day from land, sea, and space-based sensors. Daily NOAA uses Big Data to analyze and extract value from over 20 terabytes of data.

Tax Compliance
Big Data Applications can be used by tax organizations to analyze both unstructured and structured data from a variety of sources in order to identify suspicious behavior and multiple identities. This would help in tax fraud identification.

Traffic Optimization
Big Data helps in aggregating real-time traffic data gathered from road sensors, GPS devices and video cameras. The potential traffic problems in dense areas can be prevented by adjusting public transportation routes in real time.